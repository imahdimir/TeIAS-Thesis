
%%%%% Preamble %%%%%%%%%%%%%%%%%%%%%%%%%%%%%%%%%%%%%%%%%%%%%%%%%%%%%%%%%%%%%%%%
%%%%%%%%%%%%%%%%%%%%%%%%%%%%%%%%%%%%%%%%%%%%%%%%%%%%%%%%%%%%%%%%%%%%%%%%%%%%%%%


%%%%%%%%%%%%%%%% Doc Class %%%%%%%%%%%%%%%%%%%%%%%%%%%%%%%%%%

\documentclass{beamer}


%%%%%%%%%%%%%%%% Packages %%%%%%%%%%%%%%%%%%%%%%%%%%%%%%%%%%%

\usepackage{perpage} % makes foonotes per page with a commmand
\usepackage{graphicx}
\usepackage{booktabs}
\usepackage{amsmath}
\usepackage{dcolumn}
\usepackage{caption}
\usepackage{tikz}


%%%%%%%%%%%%%%%%% Theme %%%%%%%%%%%%%%%%%%%%%%%%%%%%%%%%%%%%%%

\usetheme{Frankfurt}


%%%%%%%%%%%%%%%%% Having Presentation Notes %%%%%%%%%%%%%%%%%%%

% \setbeameroption{show notes on second screen} % comment out this line to get without notes version for the purpose of distribution
% \setbeameroption{show notes} % note pages are interleaved with slides
% \setbeameroption{hide notes}


%%%%%%%%%%%%%%%%% Other Settings %%%%%%%%%%%%%%%%%%%%%%%%%%%%%%

\MakePerPage{footnote} % pkg: perpage, makes footnotes per page

\graphicspath{{./imgs/}} % pkg: graphicx, direcoty of images

\DeclareMathOperator{\E}{\mathbb{E}} % pkg: amsmath

\newcolumntype{d}[1]{D{.}{.}{#1}} % pkg: dcolumn

\setbeamertemplate{caption}[numbered] % make table and figs numbered
\setbeamertemplate{frametitle}[default][center]


%%%%%%%%%%%%%%%%% Title & Author %%%%%%%%%%%%%%%%%%%%%%%%%%%%

\title[]{\MakeUppercase{Do Institutions Follow the Herd? \\ Evidence from Iran}}
\subtitle{}

\author[]{Mahdi Mir \newline Supervised by: Mahdi Heidari}

\institute[TeIAS]{Tehran Institute for Advanced Studies}

\date[TeIAS]{\today}



%%%%%% Doc %%%%%%%%%%%%%%%%%%%%%%%%%%%%%%%%%%%%%%%%%%%%%%%%%%%%%%%%%%%%%%%%%%%%
%%%%%%%%%%%%%%%%%%%%%%%%%%%%%%%%%%%%%%%%%%%%%%%%%%%%%%%%%%%%%%%%%%%%%%%%%%%%%%%



%%%%%%%%%%%%%%%%% Doc Begin %%%%%%%%%%%%%%%%%%%%%%%%%%%%%%%%%%%
\begin{document}


%%%%%%%%%%%%%%%%%%%%%%%%%%%%%%%%%%%%%%%%%%%%%%%

\begin{frame}
    \titlepage{}
\end{frame}


\begin{frame}[label=toc]
    \frametitle{Outline}
    \tableofcontents{}
\end{frame}



\section{Introduction}



% \begin{frame}{Motivation}


% \end{frame}


\subsection{The Question}

\begin{frame}{The Question}

    \begin{itemize}
        \item The ongoing research centers around examining the trading patterns of both Institutional and Individual investors within the Tehran Stock Exchange market.
        \item Specifically, I am asking whether these institutions and individuals tend to follow momentum or anti-momentum strategies when they trade.
        \item The focus is on institutions because they're major players in the market, and naturally we expect them to make trades based on fundamentals.
        \item I am exploring momentum and anti-momentum strategies because of how they can influence market herding. This might push prices far from their real values and even lead to market bubbles.
    \end{itemize}

\end{frame}





% \begin{frame}{Conculsion}


% \end{frame}



% \begin{frame}{Related Litretature}



% \end{frame}


% \begin{frame}{Litretature Review}



% \end{frame}



\section{Data}



\begin{frame}{Data}

    \begin{enumerate}
        \item \textbf{TSE Stocks' Retruns}, \textit{Daily}
        \item \textbf{TSE Overall Index Returns (Market Return)}, \textit{Daily}
        \item \textbf{CBI Risk-Free Rate}, \textit{Monthly}
        \item \textbf{All Codal News (Letters)}, \textit{Secondly}
              \begin{itemize}
                  \item Both before 1389 (1382 - 1389) and after 1389
              \end{itemize}
        \item \textbf{TSE Individual-Institutional Trade}, \textit{Daily}
        \item \textbf{TSE Stocks' Nominal Prices}, \textit{Daily}
        \item \textbf{TSE Stocks' Market Capitalization}, \textit{Daily}
    \end{enumerate}

\end{frame}



\section{Methodology}


% explain about two stages and the model in each stage
% explain measures and defend them
% explain all variable and their sources and how they are constructed
% explain the model and the estimation method

\begin{frame}{Methodology}
    This research has two stages:

    \begin{enumerate}
        \item The first stage revolves around the categorization of company-specific news as either positive or negative. These categorizations are needed based on their influence on trading patterns of both institutions and individuals.
        \item In the subsequent stage, the labels assigned during the first phase are utilized to analyze the trading behavior of institutions and individuals within distinct and non-overlapping time frames.
    \end{enumerate}

    \par
    More details are in the following slides.

\end{frame}


\subsection{Stage 1 : News Labeling}

\begin{frame}
    \Huge
    \center
    Stage 1: News Labeling
\end{frame}


\begin{frame}{Stage 1: Why News Are Important?}

    \begin{itemize}
        \item Abundant evidence suggests that news pertaining to individual stocks, characterized as either "positive" or "negative," significantly impacts the trading choices made by investors.
        \item Any study or research design intending to analyze the trading behavior of investors of any type must take into account information related to companies.
        \item Regrettably, we currently lack a database containing company-specific news curated through assessments and opinions from expert capital market analysts.
        \item Creating such a dataset would significantly enhance the research landscape within the Tehran Stock Exchange market.
    \end{itemize}

\end{frame}


\begin{frame}{Stage 1: How I Solve The News Challenge?}

    \begin{itemize}
        \item In the first stage, I suggest a method for categorizing stock-specific news as either "positive" or "negative".
        \item The labels generated during this initial stage are then employed in the second stage as dummy variables to account for the presence of "positive" and "negative" news associated with each individual stock.
    \end{itemize}

\end{frame}


\begin{frame}{Stage 1: Source of Firm-Specific News}

    \begin{itemize}
        \item I rely on Codal for firm-specific news.
        \item I perceive every letter posted on Codal as a news item.
        \item In the context of Codal, any financial statement, report, PDF, or Excel document that is shared is regarded as a "Codal letter" within the Codal framework.
        \item Thankfully, each company-specific update on Codal is consistently linked to the respective company's ticker symbol. Each piece of content includes a ticker field.
    \end{itemize}

\end{frame}


\begin{frame}{Stage 1: How I Lebel News as Good or Bad?}
    To categorize news and systematically determine whether each stock-specific news is positive or negative, I follow this methodology:
    \begin{itemize}
        \item Using the adjusted returns data, for each firm-day pair I define a two month period prior to that day.
        \item Within this moving window, I assess the CAPM model (utilizing a rolling CAPM approach).
        \item By utilizing the CAPM betas, I compute the anticipated (expected) return for each specific firm-day.
    \end{itemize}

\end{frame}


\begin{frame}{Stage 1: How I Lebel News as Good or Bad? (Cont'd)}
    \begin{itemize}
        \item Based on the calculated expected return in the previous step, I ascertain the Abnormal Returns for each given day.
        \item I combine the computed abnormal return data with the dataset containing news articles (letters sourced from Codal).
        \item Employing a symmetric threshold, I make determinations whether news should be classified as favorable or unfavorable.
    \end{itemize}

    A few Remarks:
    \begin{itemize}
        \item The outlined steps yield a dataset comprised of trios: firm, day, and corresponding news.
        \item I haven't detailed the precise procedure and meticulous considerations within this process.
    \end{itemize}

\end{frame}


\subsection{Stage 2 : Trading Behavior}



\begin{frame}
    \Huge
    \center
    Stage 2: Trading Behavior
\end{frame}


% \begin{frame}{Stage 2: What Measure We Are Searching For?}


% \end{frame}



\begin{frame}{Stage 2: How to Measure Trading Activity?}
    To assess investors' purchasing and selling activities for stock \(i\) at time \(t\), we analyze their excess buying (\(XB_{i,t}\)) and excess selling (\(XS_{i,t}\)) of the stock individually.
    \\
    \par
    I define the excess buying and selling for the group \(\mathrm{G}\) as follows:
    \[
        \begin{aligned}
             & \mathrm{XB}_{i, t}^G=\mathrm{NB}_{i, t}^G - \E\left[\mathrm{NB}_{i, t}^G\right] \\
             & \mathrm{XS}_{i, t}^G=\mathrm{NS}_{i, t}^G - \E\left[\mathrm{NS}_{i, t}^G\right]
        \end{aligned}
    \]

    Where:
    \[
        \begin{aligned}
            \mathrm{G}                     \in & \{Institution, Individual\} \\
        \end{aligned}
    \]

\end{frame}


\begin{frame}{Stage 2: How to Measure Trading Activity?}
    The defenition of \(\mathrm{NB}_{i, t}^G\) and \(\mathrm{NS}_{i, t}^G\) are as follows:
    \[
        \begin{aligned}
            \mathrm{NB}_{i, t}^G= & \frac{\mathrm{Buy}_{i, t}^G -\operatorname{Sell}_{i, t}^G}{\mathrm{Buy}_{i, t}^G +\operatorname{Sell}_{i, t}^G} \\
            \mathrm{NS}_{i, t}^G= & \frac{\mathrm{Sell}_{i, t}^G -\operatorname{Buy}_{i, t}^G}{\mathrm{Buy}_{i, t}^G +\operatorname{Sell}_{i, t}^G}
        \end{aligned}
    \]

    Referring to the existing literature, I regrard the expeted value of
    the \(\mathrm{NB}_{i, t}^G\) and \(\mathrm{NS}_{i, t}^G\) equal to
    average of \(\mathrm{NB}_{i, t}^G\) and \(\mathrm{NS}_{i, t}^G\) for
    all stocks that investor group \(\mathrm{G}\) trades at time \(t\), respectively.

\end{frame}


\begin{frame}{Stage 2: The Model}
    We utilize fixed-effects ordinary least squares (OLS) regressions on panel data with t–statistics adjusted for panel-corrected standard errors (PCSE).
    \[
        y_{i,t} = \alpha + \beta R1 + \gamma R2 + \delta R6 + \zeta R28 + \sum_i \kappa_i * d_i + \epsilon
    \]
    where:
    \[
        y \in \{XB_{i,t}^\mathrm{Ins}, \; XS_{i,t}^\mathrm{Ins}, \; XB_{i,t}^\mathrm{Ind}, \; XS_{i,t}^\mathrm{Ind}\}
    \]

\end{frame}


\begin{frame}{Stage 2: The Model, Independent Variables}
    \[
        y_{i,t} = \alpha + \beta R1 + \gamma R2 + \delta R6 + \zeta R28 + \sum_i \kappa_i * d_i + \epsilon
    \]

    where:
    \[
        \begin{aligned}
            R1 \triangleq  & R (-1) \; \text{\small{: The one-day holding period prior to trading day.}}
            \\
            R2 \triangleq  & R(-2, -5) \; \text{\small{: The 2-5 days holding period prior to trading day.}}
            \\
            R6 \triangleq  & R(-6, -27) \; \text{\small{: The 6-27 days holding period prior to trading day.}}
            \\
            R28 \triangleq & R(-28, -119) \; \text{\small{: The 28-119 days holding period prior to trading day.}}
        \end{aligned}
    \]

    Note that:
    \begin{itemize}
        \item The holding period returns are market-adjusted returns. The adjusted return minus the market return.
        \item The market return is the TSE Overall Index. All days are working days.
    \end{itemize}

\end{frame}


\begin{frame}{Stage 2: The Model, Control Variables}
    In my analyses, I incorporate various control variables that have previously been identified as influencing investor trading behavior.
    \[
        y_{i,t} = \alpha + \beta R1 + \gamma R2 + \delta R6 + \zeta R28 + \sum_i \kappa_i * d_i + \epsilon
    \]
    The control variables are:
    \begin{itemize}
        \item To control for news, I include dummies that identify whether the contemporaneous, one-day and two-day lagged stock-specific news announcements are ‘‘good’’ or ‘‘bad’’.
    \end{itemize}

\end{frame}

\begin{frame}{Stage 2: The Model, Control Variables}

    \[
        y_{i,t} = \alpha + \beta R1 + \gamma R2 + \delta R6 + \zeta R28 + \sum_i \kappa_i * d_i + \epsilon
    \]
    The control variables are:
    \begin{itemize}
        \item To control for news, I include dummies that identify whether the contemporaneous, one-day and two-day lagged stock-specific news announcements are ‘‘good’’ or ‘‘bad’’ (6 dummies).
        \item I also include dummies to capture day-of-the-week effects. (4 dummies)
        \item And two dummy variables for ‘reference point’ effects, which equal one if the stock price is at the monthly highest or lowest level and zero otherwise.
    \end{itemize}

\end{frame}











% \section{Summary Statistics}



\section{Results}

\begin{frame}
    \Huge
    \center
    Results
\end{frame}

% provide summary statistics of data in each stage
% try to provide some graphics
% provide results of each stage
% provide results of subsetting firm to big and small

\subsection{First Stage Result}

\begin{frame}{First Stage: Results}

    \begin{itemize}
        \item The first stage of the research is to label the news as good or bad.
        \item The following table shows the distribution of news articles that have been labeled as good or bad by the explained method.
    \end{itemize}

    \center

    \begin{tabular}{lrrr}
        \toprule
        News Type & Freq.  & Percent & Cum.   \\
        \midrule
        Bad       & 25,703 & 44.78   & 44.78  \\
        Good      & 25,621 & 44.64   & 89.42  \\
        Neutral   & 6,075  & 10.58   & 100.00 \\
        \midrule
        Total     & 57,399 & 100.00  &        \\
        \bottomrule
    \end{tabular}


\end{frame}


\subsection{Second Stage Result}


\begin{frame}{Second Stage Results: Main Result}

    {\fontsize{10}{11} \selectfont

        \begin{tabular}{l*{4}{l}}
                               & \multicolumn{2}{l}{Institutions} & \multicolumn{2}{l}{Individuals}                         \\
            \cmidrule(l){2-3} \cmidrule(l){4-5}
                               & Buy                              & Sell                            & Buy       & Sell      \\
            \midrule
            \(R(-1)\)          & -1.842***                        & 1.755***                        & 0.550***  & -0.550*** \\
                               & (0.08)                           & (0.09)                          & (0.03)    & (0.03)    \\
            \(R(-2, -5)\)      & 0.046*                           & -0.083***                       & -0.006    & -0.003    \\
                               & (0.03)                           & (0.02)                          & (0.01)    & (0.01)    \\
            \(R(-6, -27)\)     & 0.053***                         & -0.049***                       & -0.028*** & 0.024***  \\
                               & (0.01)                           & (0.01)                          & (0.00)    & (0.00)    \\
            \(R(-28, -119)\)   & 0.007*                           & -0.012***                       & -0.006*** & 0.005***  \\
                               & (0.00)                           & (0.00)                          & (0.00)    & (0.00)    \\
                               &                                  &                                 &           &           \\
            Mentioned FEs      & YES                              & YES                             & YES       & YES       \\
                               &                                  &                                 &           &           \\
            N. of Unique Firms & 742                              & 742                             & 742       & 742       \\
            \(R^{2}\)          & 0.9\%                            & 0.9\%                           & 0.5\%     & 0.4\%     \\
            NObs               & 1,133,473                        & 1,133,473                       & 1,133,473 & 1,133,473 \\
            \bottomrule
            % \multicolumn{2}{c}{Standard errors in parentheses}                                     \\
            % \multicolumn{2}{c}{\textit{t-statistics} are based on panel corrected standard errors} \\
            % \multicolumn{2}{c}{*** \(p<0.01\), ** \(p<0.05\), * \(p<0.1\)}                         \\
        \end{tabular}
    }

\end{frame}

\section{Conculsion}

\begin{frame}{Main Result Interpretation}

    \begin{itemize}
        \item The findings suggest that institutions tend to go against momentum trading in the context of a one-day holding period. Conversely, for the three longer time horizons, institutions seem to adopt a momentum-oriented strategy.
        \item On the other hand, the results reveal that individuals engage in momentum trading within a one-day holding period but shift towards an anti-momentum approach for longer horizons.
        \item Overall, the trading behaviors of institutional and individual traders exhibit a notable counteractive nature.
    \end{itemize}

\end{frame}


\begin{frame}{Conculsion}

    \begin{itemize}
        \item The present study introduces an approach to categorize stock-specific news into positive and negative classifications.
        \item In the context of a one-day timeframe, institutional traders appear to adopt an anti-momentum approach, whereas they seem to embrace a momentum strategy over longer periods.
        \item The findings highlight systematic distinctions between institutions and individuals regarding their responses to historical price trends, as well as their varying degrees of adherence to momentum and contrarian strategies.
    \end{itemize}

\end{frame}







% \begin{frame}{Second Stage Results: Main Result by Firm Size}
%     % this slide was a bit wide so I tried to fit in the data by abbreviating the columns
%     % fixed effects have been used but not shown in the slides table but they are present in the text.
%     % I removed individuals because their coefficients does not vary I have to change them for now I empheasize on funds

%     {\fontsize{10}{10} \selectfont

%         \begin{tabular}{l*{4}{l}}
%                                & \multicolumn{4}{l}{Institutions}                                                           \\
%             \cmidrule(l){2-5}

%                                & \multicolumn{2}{l}{Large Stocks} & \multicolumn{2}{l}{Small Stocks}                        \\
%             \cmidrule(l){2-3} \cmidrule(l){4-5}
%                                & Buy                              & Sell                             & Buy       & Sell     \\
%             \midrule
%             \(R(-1)\)             & -3.215***                        & 3.137***                         & -0.705*** & 0.662*** \\
%                                   & (0.10)                           & (0.12)                           & (0.05)    & (0.05)   \\
%             \(R(-2, -5)\)         & 0.175***                         & -0.190***                        & -0.048*   & 0.022    \\
%                                   & (0.04)                           & (0.04)                           & (0.03)    & (0.02)   \\
%             \(R(-6, -27)\)        & 0.063***                         & -0.057***                        & 0.017     & 0.002    \\
%                                   & (0.02)                           & (0.02)                           & (0.02)    & (0.01)   \\
%             \(R(-28, -119)\)      & -0.003                           & -0.000                           & 0.003     & -0.000   \\
%                                & (0.00)                           & (0.00)                           & (0.01)    & (0.00)   \\
%                                &                                  &                                  &           &          \\
%             Mentioned FEs      & YES                              & YES                              & YES       & YES      \\
%                                &                                  &                                  &           &          \\
%             N. of Unique Firms & 646                              & 646                              & 411       & 411      \\
%             \(R^{2}\)          & 2.2\%                            & 2.1\%                            & 0.2\%     & 0.2\%    \\
%             NObs               & 390,655                          & 390,655                          & 256,712   & 256,712  \\
%             \bottomrule
%             % \multicolumn{2}{c}{Standard errors in parentheses}                                     \\
%             % \multicolumn{2}{c}{\textit{t-statistics} are based on panel corrected standard errors} \\
%             % \multicolumn{2}{c}{*** \(p<0.01\), ** \(p<0.05\), * \(p<0.1\)}                         \\
%         \end{tabular}
%     }

% \end{frame}






% replication package url on github on the Thank you page


\section{Appendix}



























%%%%%%%%%%%%%%%%% Doc End %%%%%%%%%%%%%%%%%%%%%%%%%%%%%%%%%%%

\end{document}